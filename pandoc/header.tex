\usepackage{xeCJK}
\usepackage{geometry}
\usepackage{fancyhdr}
\usepackage{titlesec}
\usepackage{tcolorbox}
\usepackage{listings}
\usepackage{xcolor}
\usepackage{fontspec}
\usepackage{microtype}
\usepackage[hidelinks]{hyperref}
\usepackage{booktabs}
\usepackage{caption}
\usepackage{float}
\usepackage{graphicx}

% ページサイズとマージンの設定(A4、技術書らしい余白)
\geometry{
  a4paper,
  left=2.5cm,
  right=2cm,
  top=2.5cm,
  bottom=2.5cm,
  headheight=15pt,
  headsep=20pt,
  footskip=30pt
}

% フォント設定(技術書らしいフォント、フォールバック対応)
\setCJKmainfont[
  BoldFont={Noto Sans CJK JP Bold},
  FallBack={DejaVu Serif}
]{Noto Sans CJK JP}
\setmainfont[
  Ligatures=TeX,
  BoldFont={Liberation Serif Bold},
  ItalicFont={Liberation Serif Italic},
  FallBack={DejaVu Serif}
]{Liberation Serif}
\setsansfont[
  Ligatures=TeX,
  BoldFont={Liberation Sans Bold},
  FallBack={DejaVu Sans}
]{Liberation Sans}
\setmonofont[
  Scale=0.9,
  BoldFont={Liberation Mono Bold},
  FallBack={DejaVu Sans Mono}
]{Liberation Mono}

% ヘッダー・フッターの設定
\pagestyle{fancy}
\fancyhf{}
\fancyhead[LE,RO]{\thepage}
\fancyhead[LO]{\rightmark}
\fancyhead[RE]{\leftmark}
\renewcommand{\headrulewidth}{0.4pt}

% セクション見出しのスタイル
\titleformat{\chapter}[display]
  {\normalfont\Large\bfseries\color{black}}
  {\chaptertitlename\ \thechapter}{20pt}{\Huge}
\titleformat{\section}
  {\normalfont\large\bfseries\color{blue!80!black}}
  {\thesection}{1em}{}
\titleformat{\subsection}
  {\normalfont\normalsize\bfseries\color{blue!60!black}}
  {\thesubsection}{1em}{}

% コードブロックの設定
\tcbuselibrary{listings,skins,breakable}

% シンタックスハイライト用の色定義
\definecolor{codebg}{RGB}{248,248,248}
\definecolor{codeframe}{RGB}{232,232,232}
\definecolor{comment}{RGB}{106,153,85}
\definecolor{keyword}{RGB}{86,156,214}
\definecolor{string}{RGB}{206,145,120}
\definecolor{number}{RGB}{181,206,168}

% lstlisting環境の基本設定
\lstset{
  basicstyle=\small\ttfamily,
  breaklines=true,
  breakatwhitespace=true,
  columns=flexible,
  keepspaces=true,
  showstringspaces=false,
  tabsize=2,
  frame=single,
  backgroundcolor=\color{codebg},
  rulecolor=\color{codeframe},
  numberstyle=\tiny\color{gray},
  keywordstyle=\color{keyword}\bfseries,
  commentstyle=\color{comment}\itshape,
  stringstyle=\color{string},
  numberstyle=\color{number},
  xleftmargin=10pt,
  xrightmargin=10pt,
  framexleftmargin=8pt,
  framexrightmargin=8pt,
  captionpos=t,
  aboveskip=15pt,
  belowskip=15pt
}

% 各言語のシンタックスハイライト設定
\lstdefinelanguage{JavaScript}{
  keywords={abstract, arguments, boolean, break, byte, case, catch, char, class, const, continue, debugger, default, delete, do, double, else, enum, eval, export, extends, false, final, finally, float, for, function, goto, if, implements, import, in, instanceof, int, interface, let, long, native, new, null, package, private, protected, public, return, short, static, super, switch, synchronized, this, throw, throws, transient, true, try, typeof, var, void, volatile, while, with, yield},
  keywordstyle=\color{keyword}\bfseries,
  ndkeywords={class, export, boolean, throw, implements, import, this},
  ndkeywordstyle=\color{blue!80!black}\bfseries,
  identifierstyle=\color{black},
  sensitive=false,
  comment=[l]{//},
  morecomment=[s]{/*}{*/},
  commentstyle=\color{comment}\ttfamily,
  stringstyle=\color{string}\ttfamily,
  morestring=[b]',
  morestring=[b]"
}

\lstdefinelanguage{TypeScript}{
  keywords={abstract, arguments, boolean, break, byte, case, catch, char, class, const, continue, debugger, default, delete, do, double, else, enum, eval, export, extends, false, final, finally, float, for, function, goto, if, implements, import, in, instanceof, int, interface, let, long, native, new, null, package, private, protected, public, return, short, static, super, switch, synchronized, this, throw, throws, transient, true, try, typeof, var, void, volatile, while, with, yield, type, as, readonly, keyof, infer, namespace, declare},
  keywordstyle=\color{keyword}\bfseries,
  ndkeywords={class, export, boolean, throw, implements, import, this, type, interface, readonly},
  ndkeywordstyle=\color{blue!80!black}\bfseries,
  identifierstyle=\color{black},
  sensitive=false,
  comment=[l]{//},
  morecomment=[s]{/*}{*/},
  commentstyle=\color{comment}\ttfamily,
  stringstyle=\color{string}\ttfamily,
  morestring=[b]',
  morestring=[b]"
}

% インラインコードの設定
\newcommand{\passthrough}[1]{#1}

% 表の設定
\captionsetup[table]{
  position=top,
  skip=5pt,
  font=small,
  labelfont=bf
}

% 図の設定
\captionsetup[figure]{
  position=bottom,
  skip=5pt,
  font=small,
  labelfont=bf
}

% リンクの色設定
\hypersetup{
  colorlinks=true,
  linkcolor=blue!80!black,
  urlcolor=blue!80!black,
  citecolor=blue!80!black,
  filecolor=blue!80!black
}

% 長い単語の改行設定(英語の文字はみ出し対策)
\emergencystretch=3em
\tolerance=9999
\hbadness=9999

% マイクロタイポグラフィの有効化
\microtypesetup{
  protrusion=true,
  expansion=true,
  kerning=true,
  spacing=true
}

% 日本語と英語の間のスペース調整
\xeCJKsetup{CJKglue={\hskip 0pt plus 0.08\baselineskip}}

% 章の開始ページを奇数ページに固定しない
\let\cleardoublepage\clearpage
